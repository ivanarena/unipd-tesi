% !TEX encoding = UTF-8
% !TEX TS-program = pdflatex
% !TEX root = ../tesi.tex

%**************************************************************
\chapter{Processi e metodologie}
\label{cap:processi-metodologie}
%**************************************************************

\intro{In questo capitolo vengono descritte le modalità con cui si è svolto lo stage e le tecnologie utilizzate.}\\

%**************************************************************
\section{Organizzazione del lavoro}

Il lavoro è stato svolto in parte a tempo parziale ed in parte a tempo pieno, data la concomitanza con il progetto di laboratorio del corso di Ingegneria del Sofware. 

Il piano di lavoro dello stage individuava tre periodi principali:

\begin{itemize}
    \item Un periodo di formazione sulle tecnologie utilizzate, della durata totale prevista di 80 ore e suddiviso in tre macro-categorie relative rispettivamente all'\acrshort{sso}, all'\acrshort{ssi} e a FreeIPA, Linux \acrshort{pam};
    \item Un periodo riservato alle integrazioni software, della durata totale prevista di 200 ore;
    \item Un periodo di stesura di documentazione relativa allo stage e alla tesi, della durata totale prevista di 20 ore.
\end{itemize}




