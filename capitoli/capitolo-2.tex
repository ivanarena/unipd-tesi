% !TEX encoding = UTF-8
% !TEX TS-program = pdflatex
% !TEX root = ../tesi.tex

%**************************************************************
\chapter{Processi e metodologie}
\label{cap:processi-metodologie}
%**************************************************************

\intro{In questo capitolo vengono descritte le modalità con cui si è svolto lo stage e le tecnologie utilizzate.}\\

%**************************************************************
\section{Organizzazione del lavoro}

Il lavoro è stato suddiviso in n periodi.

Nel primo ho fatto xyz

\section{Tecnologie utilizzate}
\subsection{FreeIPA}
FreeIPA è una soluzione open-source gratuita (GNU General Public License) di gestione dell'identità e dell'accesso per ambienti di rete basati su Linux/UNIX, originariamente sviluppato dalla comunità Fedora ed ora supportato da diverse organizzazioni, tra cui Red Hat e la FreeIPA Foundation. Consiste in un insieme di servizi integrati, che consentono di centralizzare l'autenticazione, l'autorizzazione e la gestione degli utenti e delle risorse in un'organizzazione.

FreeIPA è progettato per semplificare la gestione dell'identità e dell'accesso in ambienti di rete complessi, con molti utenti e computer. Consente agli amministratori di gestire facilmente l'accesso degli utenti a risorse e applicazioni, di delegare i privilegi di amministrazione e di definire ed applicare politiche di sicurezza coerenti in tutta la rete, come, ad esempio, limitare l'accesso alle risorse in base al ruolo dell'utente. Per fare ciò, mette a disposizione, oltre che agli strumenti della CLI, un'interfaccia utente web intuitiva per la gestione degli utenti, dei gruppi e delle risorse della rete. Inoltre, FreeIPA è altamente scalabile e può essere distribuito su più server per gestire grandi reti.

Per l'autenticazione degli utenti, FreeIPA utilizza il protocollo Kerberos: gli utenti possono accedere alle risorse della rete utilizzando le loro credenziali Kerberos, senza dover inserire le password ogni volta.
Per archiviare e gestire le informazioni sugli utenti, i gruppi e le risorse della rete, invece, utilizza il server di directory open-source 389 Directory Server, il quale offre funzionalità avanzate di ricerca, replica e sincronizzazione.

FreeIPA supporta l'autenticazione SSO tramite il protocollo SAML (Security Assertion Markup Language), ciò significa che gli utenti possono accedere a più applicazioni utilizzando le stesse credenziali di accesso.

\section{LXC}
LXC è l'acronimo di Linux Containers, un sistema di virtualizzazione basato sul kernel Linux che consente di eseguire più sistemi operativi isolati su una singola macchina host. A differenza della virtualizzazione completa, in cui ogni sistema operativo guest ha accesso all'intero hardware dell'host, 

Utilizza la virtualizzazione basata sui contenitori, in cui ogni sistema operativo guest condivide le risorse hardware dell'host.
La condivisione del kernel fa sì che i container siano molto leggeri e veloci e che abbiano un overhead di risorse molto basso rispetto ad altre tecnologie di virtualizzazione.

LXC fornisce un'interfaccia di riga di comando per la gestione dei container, che consente di creare, avviare, fermare, eliminare e gestire i container in modo semplice ed efficiente. Inoltre, supporta la creazione di immagini di container, che possono essere utilizzate per creare nuovi container in modo rapido e semplice.

Durante l'attività di stage ho utilizzato principalmente container basati su immagini CentOS (Community Enterprise Operating System), una distribuzione Linux basata su Red Hat Enterprise Linux (RHEL) particolarmente adatta all'uso in ambiente server, che offre una vasta gamma di funzionalità e strumenti per gestire un'infrastruttura IT.

\section{SSH}
Secure Shell (SSH) è un protocollo di rete crittografato utilizzato per la gestione sicura di dispositivi di rete e per l'accesso remoto a sistemi informatici. Il protocollo SSH fornisce un canale di comunicazione sicuro tra due dispositivi, garantendo l'integrità, la riservatezza e l'autenticità delle informazioni trasmesse.

L'autenticazione avviene attraverso l'uso di chiavi pubbliche e private: in questo metodo, un'entità che desidera accedere a un sistema remoto genera una coppia di chiavi, una pubblica e una privata; la chiave pubblica viene fornita al sistema remoto, mentre la chiave privata viene conservata dall'entità; quando l'entità si connette al sistema remoto, la chiave privata viene utilizzata per autenticare l'entità.

Inoltre, SSH utilizza la crittografia per proteggere i dati trasferiti tra i dispositivi. In particolare, il protocollo utilizza la crittografia a chiave simmetrica per proteggere i dati durante la trasmissione, e la crittografia a chiave pubblica per autenticare le parti coinvolte.