% !TEX encoding = UTF-8
% !TEX TS-program = pdflatex
% !TEX root = ../tesi.tex

%**************************************************************
\chapter{Conclusioni}
\label{cap:conclusioni}
%**************************************************************

\intro{In quest'ultimo capitolo vengono riportate le conclusioni e gli esiti dell'attività di stage}\\


%**************************************************************

\section{Raggiungimento degli obiettivi}

L'attività è stata svolta quasi totalmente in linea con la pianificazione prevista: non ci sono stati ritardi di alcun tipo ed il primo periodo, quello di studio delle tecnologie, ha richiesto meno tempo di quanto preventivato, consentendomi, così, di approfondire ulteriormente lo sviluppo delle soluzioni trovate nelle fasi successive.
Ho completato tutti gli obiettivi richiesti con successo, ad inclusione di quelli desiderabili e opzionali: dopo aver implementato con successo l'\acrshort{sso} di Monokee in una macchina \acrshort{centos} utilizzando FreeIPA, ho prodotto la documentazione relativa, illustrando le procedure da seguire per replicare l'integrazione su altre macchine.

%**************************************************************
\section{Conoscenze acquisite}

Grazie allo stage con \myAzienda{} mi sono addentrato in un campo dell'informatica che poco conoscevo, quello della sicurezza. Ho avuto modo di conoscere i concetti e le tecnologie più significative del momento presente in ambito di identità digitale, come la \acrshort{ssi}, il \acrshort{sso} ed alcuni dei protocolli di autenticazione ed autorizzazione più diffusi, apprendendo, anzitutto, cosa significa creare e gestire un'identità digitale e quali sono i rischi di sicurezza legati ad essa. Oltre ad una già ampia formazione teorica, ho avuto anche la possibilità di migliorare le mie competenze in ambito di sistemi UNIX, entrando a contatto con parti di codice che cambiano direttamente il comportamento del sistema operativo, come i moduli \acrshort{pam} e l'\acrshort{ssh}.
Inoltre, ho imparato a creare dei container \acrshort{lxc} dalle immagini messe a disposizione e, successivamente, a configurare un server di Identity and Access Management, quale FreeIPA, modificando anche, in alcuni casi, manualmente dei file di sistema.
Infine, ho messo in atto le conoscenze acquisite nella prima fase dell'attività, in particolare quelle riguardanti il funzionamento di \acrshort{oauth2} ed \acrshort{oidc}, per implementare il \acrfull{poc} richiesto. 

%**************************************************************
\section{Valutazione personale}

Dopo circa trecento ore passate al fianco del team di \myAzienda{} sono convinto di aver acquisito delle conoscenze e delle competenze, non strettamente tecniche, fondamentali per il mio ingresso prossimo nell'industria: questa esperienza, che costituisce la mia prima nell'ambito del percorso che mi appartiene, quello dell'informatica, mi ha permesso di affrontare personalmente e toccare con mano le sfide, i problemi, le metodologie ed i traguardi propri della realtà delle aziende informatiche.

A partire dalla comunicazione, dall'organizzazione e dalla gestione del tempo e delle risorse, arrivando poi agli effettivi processi risolutivi e di sviluppo, sento di aver ricevuto un contributo signifcativo e di essermi messo alla prova, applicandomi al meglio in un ambiente a me quasi del tutto sconosciuto, al di fuori della mia zona di comfort.
Ora, al momento della stesura di questo documento, ripercorrendo ciò che ho fatto durante questa attività di stage, mi rendo conto ancora meglio del valore che essa ha avuto ed ha per me e per la mia carriera.
