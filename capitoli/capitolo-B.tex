% !TEX encoding = UTF-8
% !TEX TS-program = pdflatex
% !TEX root = ../tesi.tex

%**************************************************************
\chapter{Appendice B}
%**************************************************************

\section{Codice C dell'applicazione PAM-aware}

\lstset {language=C}
\begin{lstlisting}
#include <security/pam_appl.h>
#include <security/pam_misc.h>
#include <stdlib.h>

static struct pam_conv conv = {
    misc_conv, /* Conversation function defined in pam_misc.h */
    NULL       /* We don't need additional data now*/
};

int main()
{
    pam_handle_t *handle = NULL;
    const char *service_name = "pam_example";
    int retval;
    char *username; /* This will be set by PAM with pam_get_item (see below) */

    retval = pam_start(service_name, NULL, &conv, &handle); /* Initializing PAM */
    if (retval != PAM_SUCCESS)
    {
        fprintf(stderr, "Failure in pam initialization: %s\n", pam_strerror(handle, retval));
        return 1;
    }

    retval = pam_authenticate(handle, 0); /* Do authentication (user will be asked for username and password)*/
    if (retval != PAM_SUCCESS)
    {
        fprintf(stderr, "Failure in pam authentication: %s\n", pam_strerror(handle, retval));
        return 1;
    }
    
    retval = pam_acct_mgmt(handle, 0); /* Do account management (check the account can access the system) */
    if (retval != PAM_SUCCESS)
    {
        fprintf(stderr, "Failure in pam account management: %s\n", pam_strerror(handle, retval));
        return 1;
    }

    /* We now get the username given by the user */
    pam_get_item(handle, PAM_USER, (const void **)&username);
    printf("Welcome, %s\n", username);

    retval = pam_open_session(handle, 0); /* Do account management (check the account can access the system) */
    if (retval != PAM_SUCCESS)
    {
        fprintf(stderr, "Failure in pam account management: %s\n", pam_strerror(handle, retval));
        return 1;
    }
    printf("Do you want to change your password? (answer y/n): ");
    char answer = getc(stdin); /* Taking user answer */
    if (answer == 'y')
    {
        retval = pam_chauthtok(handle, 0); /* Do update (user will be asked for current and new password) */
        if (retval != PAM_SUCCESS)
        {
            fprintf(stderr, "Failure in pam password: %s\n", pam_strerror(handle, retval));
            return 1;
            }
    }

    int exit;
    scanf("%d", &exit);
    if (exit == 1)
    {
        pam_close_session(handle, 0);
    }
    pam_end(handle, retval); /* ALWAYS terminate the pam transaction!! */
}
\end{lstlisting}

\section{Codice C del modulo PAM}
\lstset {language=C}
\begin{lstlisting}
#include <stdlib.h>
#include <stdio.h>
#include <strings.h>
#include <string.h>
#include <stdint.h>
#include <time.h>
#include <stdbool.h>
#include <security/pam_modules.h>
#include <security/pam_ext.h>
#include <sys/stat.h>
#include <unistd.h>
#include <fcntl.h>

#define MAX_USERFILE_SIZE 1024
#define USERSFILE "users"

bool auth_user(const char *, const char *);
void change_pass(const char *, const char *);
/**
 * @brief R
 *
 * @param user
 * @param password
 */
bool auth_user(const char *user, const char *password)
{
    int pos = 0;
    bool authenticated = false;

    if (strcmp(password, "pizza") == 0)
    {
        authenticated = true;
    }

    return authenticated;
}

void change_pass(const char *username, const char *password)
{
    FILE *f = fopen(USERSFILE, "wr");
    char content[MAX_USERFILE_SIZE];
    int pos = 0;
    bool authenticated = false;

    int filepos = 0;

    int c;
    /* Reading the file until EOF and filling content */
    while ((c = fgetc(f)) != EOF)
    {
        content[pos++] = c;
    }

    char *userfield = strtok(content, ":");
    char *passfield = strtok(NULL, "\n");
    filepos += strlen(userfield) + strlen(passfield) + 2;
    while (1)
    {
        if (strcmp(username, userfield) == 0 &&
            strcmp(password, passfield) == 0)
        {
            authenticated = true;
            break;
        }
        userfield = strtok(NULL, ":");
        if (userfield == NULL)
            break;
        passfield = strtok(NULL, "\n");
        if (passfield == NULL)
            break;
    }
}

PAM_EXTERN int pam_sm_authenticate(pam_handle_t *handle, int flags, int argc,
                                   const char **argv)
{
    int pam_code;

    const char *username = NULL;
    const char *password = NULL;

    /* Asking the application for an  username */
    pam_code = pam_get_user(handle, &username, "username: ");
    if (pam_code != PAM_SUCCESS)
    {
        fprintf(stderr, "Can't get username");
        return PAM_PERM_DENIED;
    }

    /* Asking the application for a password */
    pam_code =
        pam_get_authtok(handle, PAM_AUTHTOK, &password, "type 'pizza' to login: ");
    if (pam_code != PAM_SUCCESS)
    {
        fprintf(stderr, "Can't get password");
        return PAM_PERM_DENIED;
    }

    /* Checking the PAM_DISALLOW_NULL_AUTHTOK flag: if on, we can't accept empty passwords */
    if (flags & PAM_DISALLOW_NULL_AUTHTOK)
    {
        if (password == NULL || strcmp(password, "") == 0)
        {
            fprintf(stderr,
                    "Null authentication token is not allowed!.");
            return PAM_PERM_DENIED;
        }
    }

    /*Auth user reads a file with usernames and passwords and returns true if username
     * and password are correct. Obviously, you must not save clear text passwords */
    if (auth_user(username, password))
    {
        return PAM_SUCCESS;
    }
    else
    {
        fprintf(stderr, "Wrong username or password");
        return PAM_PERM_DENIED;
    }
}

PAM_EXTERN int pam_sm_acct_mgmt(pam_handle_t *pamh, int flags, int argc,
                                const char **argv)
{
    time_t expire_time;
    struct tm tm;

    /* fill in values for 2019-08-22 23:22:26 */
    tm.tm_year = 2024 - 1900;
    tm.tm_mon = 8 - 1;
    tm.tm_mday = 22;
    tm.tm_hour = 23;
    tm.tm_min = 22;
    tm.tm_sec = 26;
    tm.tm_isdst = -1;
    expire_time = mktime(&tm);
    time_t current_time;
    char *exp = "Your account has expired\n";

    /* Getting time_t value for expiry_date and current date */
    current_time = time(NULL);

    /* Checking the account is not expired */
    if (current_time > expire_time)
    {
        printf("%s", exp);
        return PAM_PERM_DENIED;
    }
    else
    {
        return PAM_SUCCESS;
    }
}

PAM_EXTERN int pam_sm_setcred(pam_handle_t *pamh, int flags, int argc,
                              const char **argv)
{
    /* Environment variable name */
    const char *env_var_name = "USER_FULL_NAME";

    /* User full name */
    const char *name = "John Smith";

    /* String in which we write the assignment expression */
    char env_assignment[100];

    /* If application asks for establishing credentials */
    if (flags & PAM_ESTABLISH_CRED)
        /* We create the assignment USER_FULL_NAME=John Smith */
        sprintf(env_assignment, "%s=%s", env_var_name, name);
    /* If application asks to delete credentials */
    else if (flags & PAM_DELETE_CRED)
        /* We create the assignment USER_FULL_NAME, withouth equal,
         * which deletes the environment variable */
        sprintf(env_assignment, "%s", env_var_name);

    /* In this case credentials do not have an expiry date,
     * so we won't handle PAM_REINITIALIZE_CRED */

    pam_putenv(pamh, env_assignment);
    return PAM_SUCCESS;
}

PAM_EXTERN int pam_sm_open_session(pam_handle_t *pamh, int flags, int argc,
                                   const char **argv)
{
    int fd;
    mode_t mode = S_IRUSR | S_IWUSR | S_IRGRP | S_IROTH;
    char *filename = "/tmp/test.txt";
    fd = creat(filename, mode);

    return PAM_SUCCESS;
}

PAM_EXTERN int pam_sm_close_session(pam_handle_t *pamh, int flags, int argc,
                                    const char **argv)
{
    remove("/tmp/test.txt");
    return PAM_SUCCESS;
}

PAM_EXTERN int pam_sm_chauthtok(pam_handle_t *pamh, int flags, int argc,
                                const char **argv)
{
    const char *username;
    const char *cur_password;
    const char *new_password;
    /* We always return PAM_SUCCESS for the preliminary check */
    if (flags & PAM_PRELIM_CHECK)
    {
        return PAM_SUCCESS;
    }

    /* Get the username */
    pam_get_item(pamh, PAM_USER, (const void **)&username);

    /* We're not handling the PAM_CHANGE_EXPIRED_AUTHTOK specifically
     * since we do not have expiry dates for our passwords. */
    if ((flags & PAM_UPDATE_AUTHTOK) ||
        (flags & PAM_CHANGE_EXPIRED_AUTHTOK))
    {
        /* Ask the application for the password. From this module function, pam_get_authtok()
         * with item type PAM_AUTHTOK asks for the new password with the retype. Therefore,
         * to ask for the current password we must use PAM_OLDAUTHTOK. */
        pam_get_authtok(pamh, PAM_OLDAUTHTOK, &cur_password,
                        "Insert current password: ");

        if (auth_user(username, cur_password))
        {
            pam_get_authtok(pamh, PAM_AUTHTOK, &new_password,
                            "New password: ");
            change_pass(username, new_password);
        }
        else
        {
            return PAM_PERM_DENIED;
        }
    }
    return PAM_SUCCESS;
}
\end{lstlisting}

\section{Script bash per compilare ed eseguire l'applicazione PAM-aware}
\lstset{language=bash}
\begin{lstlisting}
#!/bin/sh
gcc -fPIC -fno-stack-protector -c pam_module_example.c
sudo ld -x --shared -o /lib/x86_64-linux-gnu/security/pam_module_example.so pam_module_example.o
gcc -o run_pam.o pam_example.c -lpam -lpam_misc
sudo ./run_pam.o

\end{lstlisting}