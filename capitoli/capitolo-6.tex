% !TEX encoding = UTF-8
% !TEX TS-program = pdflatex
% !TEX root = ../tesi.tex

%**************************************************************
\chapter{Implementazione e documentazione}
\label{cap:implementazione-documentazione}
%**************************************************************

\intro{In questo capitolo viene illustrato il processo di implementazione della soluzione trovata e la stesura della documentazione relativa.}


\section{Configurazione Monokee}
All'interno dell'ambiente di test di Monokee, tramite l'interfaccia web, ho creato una nuova applicazione OAuth2 ed un nuovo OpenID Connect provider, che fornisse gli end-point per l'autenticazione via OICD.


\section{Configurazione FreeIPA}
Da FreeIPA, ho proceduto a configurare un nuovo Identity Provider server, tramite la sezione della GUI di cui sopra, inserendo tutte i metadati richiesti, facendo riferimento all'applicazione OAuth2 creata su Monokee e agli end-point forniti dall'OpenID provider configurato precedentemente.

Successivamente, ho creato un utente FreeIPA che utilizzasse come unico metodo di autenticazione quella tramite Identity Provider esterno (\emph{External IdP}), scegliendo Monokee come IdP e come identificatore il mio indirizzo e-mail istituzionale, già associato al mio account Monokee, per poter eseguire l'accesso via SSO con le mie credenziali.  
\section{Testing}
Configurata correttamente l'infrastruttura di autenticazione sia su Monokee che su FreeIPA, ho proceduto a testarla seguendo le indicazioni della documentazione relativa.

\section{Troubleshooting}

\section{Sviluppi futuri}
\section{Documentazione}

L'azienda ha richiesto la redazione di una guida che illustrare il processo di configurazione del server FreeIPA e dei sistemi di Monokee per l'integrazione del Single Sign-On sulle macchine UNIX, per fornire una base documentativa per facilitare le future progressioni e sperimentazioni relative. Ho steso tale documentazione in formato Markdown, versionando il codice sul repository GitHub aziendale fornito da \myAzienda.