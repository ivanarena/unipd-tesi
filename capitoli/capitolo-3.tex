% !TEX encoding = UTF-8
% !TEX TS-program = pdflatex
% !TEX root = ../tesi.tex

%**************************************************************
\chapter{Studio tecnologico}
\label{cap:studio-tecnologico}
%**************************************************************

\intro{In questo capitolo vengono illustrati nel dettaglio i concetti e le tecnologie utlizzate.}\\

%**************************************************************
\section{Single Sign-On}

Il Single Sign On (SSO) è una tecnologia che consente agli utenti di accedere a più applicazioni e servizi utilizzando un'unica identità di accesso. In pratica, l'utente inserisce le proprie credenziali di accesso una sola volta e successivamente può accedere a tutte le applicazioni e servizi che supportano l'SSO senza dover inserire nuovamente le credenziali, alternativamente a come accade con l'autenticazione tradizionale. Ciò può risultare particolarmente vantaggioso se l'utente necessita di accedere a molte applicazioni o servizi diversi.

L'SSO funziona attraverso l'utilizzo di un'autorità di autenticazione centralizzata, chiamata Identity Provider (IdP). L'IdP autentica l'utente e fornisce un token di sicurezza che contiene le informazioni sull'utente e sui servizi a cui ha accesso. Questo token può essere utilizzato per accedere a tutti i servizi che supportano tale sistema.

Per utilizzare l'SSO, le applicazioni e i servizi devono supportare uno dei protocolli SSO standard, come SAML (Security Assertion Markup Language) o OpenID Connect. Questi protocolli definiscono il modo in cui le informazioni di autenticazione dell'utente vengono trasmesse tra le diverse applicazioni e servizi.

L'SSO offre, dunque, numerosi vantaggi, tra cui una maggiore comodità per gli utenti, una maggiore sicurezza attraverso l'utilizzo di token di sicurezza a breve termine e una maggiore efficienza nella gestione delle identità e delle autorizzazioni degli utenti. Tuttavia, l'SSO richiede una pianificazione e una configurazione adeguata per garantire la sicurezza e la protezione dei dati degli utenti.

\section{Self-Sovereign Identity}

L'SSI (Self-Sovereign Identity) è un nuovo approccio alla gestione delle identità digitali che consente agli utenti di possedere, controllare e condividere le proprie informazioni di identità in modo sicuro e privato. A differenza dei sistemi di identità tradizionali, in cui le informazioni di identità sono conservate in modo centralizzato da terze parti, l'SSI consente agli utenti di essere i proprietari esclusivi dei propri dati di identità digitali.

L'SSI si basa sulla tecnologia blockchain, che consente di creare registri distribuiti di informazioni sicure e immutabili. In tal modo, le informazioni di identità degli utenti vengono conservate in modo decentralizzato e sicuro, senza la necessità di un'autorità centralizzata di controllo.

Per utilizzare l'SSI, gli utenti creano un'identità digitale che include le informazioni di identità necessarie, come nome, indirizzo e informazioni di contatto. Questa identità digitale viene conservata sulla blockchain e protetta da una chiave privata unica, che solo l'utente possiede.

Gli utenti possono utilizzare la propria identità digitale SSI per accedere a servizi online e condividere le proprie informazioni di identità solo con le parti che desiderano. Questo viene fatto attraverso l'utilizzo di un protocollo di scambio di informazioni sicuro e decentralizzato, chiamato DID (Decentralized Identifier).

L'SSI offre numerosi vantaggi, tra cui un maggiore controllo e privacy per gli utenti rispetto ai sistemi di identità tradizionali, una maggiore sicurezza attraverso l'utilizzo della tecnologia blockchain e una maggiore efficienza nella gestione delle identità digitali. Tuttavia, è ancora una tecnologia emergente e richiede una maggiore adozione e sviluppo per diventare un approccio mainstream alla gestione delle identità digitali.


\section{OAuth2}
OAuth2 è un protocollo di autorizzazione che consente a un'applicazione di accedere alle risorse di un utente senza richiedere le credenziali dell'utente - e, di conseguenza, senza memorizzarle - nato per assicurare l'accesso sicuro e controllato ai dati di un utente da parte di applicazioni di terze parti.

Funziona attraverso una serie di flussi di autorizzazione, in cui l'utente concede l'autorizzazione all'applicazione per accedere alle sue risorse. L'applicazione, a sua volta, ottiene un token di accesso che può essere utilizzato per accedere alle risorse dell'utente.

Il protocollo OAuth2 è utilizzato da molte grandi piattaforme online come Google, Facebook e Twitter ed è diventato, di fatto, uno standard nei servizi cloud, nei social network, nei servizi di pagamento online e in molti altri contesti.
\section{OpenID Connect}
OpenID Connect (OIDC) è un protocollo di autenticazione basato su OAuth2, utilizzato per l'autenticazione degli utenti in applicazioni web e mobile. Progettato per risolvere il problema dell'autenticazione sicura e decentralizzata in applicazioni di terze parti, consente agli utenti di utilizzare l'SSO per accedere a diverse applicazioni, senza, quindi, dover creare un nuovo account per ogni applicazione, bensì delegando l'autenticazione ad un provider esterno (OpenID Provider).

OIDC fornisce un framework standard per l'autenticazione basata su JSON Web Tokens (JWT), in cui l'utente viene autenticato una sola volta e poi viene rilasciato un token di accesso contenente le informazioni di base dell'utente, come l'identificatore univoco, il nome e l'email, che può essere utilizzato per accedere alle risorse protette.

Questo protocollo è stato adottato da molte grandi piattaforme online, tra cui Google, Microsoft e Amazon ed è anche supportato da molte librerie di sviluppo, caratteristica che lo rende semplice da implementare per gli sviluppatori.

\section{Linux PAM}
Linux PAM (Pluggable Authentication Modules) è un framework di autenticazione per i sistemi operativi Linux e UNIX che consente di configurare diversi metodi di autenticazione, come quella tramite password, a due fattori, basata su token, biometrica, ecc.

Utilizzato in una vasta gamma di applicazioni e servizi, tra cui il sistema di login del sistema operativo ed il server SSH, il framework di PAM è composto da una serie di moduli, ognuno dei quali implementa una particolare funzionalità di autenticazione. I moduli PAM sono progettati per essere "pluggable", ovvero possono essere facilmente sostituiti o aggiunti senza dover modificare il codice sorgente del sistema operativo.

L'architettura modulare di PAM consente di creare una catena di moduli, in cui ciascuno dei quali può verificare una parte dell'identità dell'utente. Ad esempio, un modulo può verificare la password dell'utente, mentre un altro può verificare il certificato del client. Se uno qualsiasi dei moduli nella catena fallisce, l'intero processo di autenticazione viene interrotto. Inoltre, è possibile sviluppare dei moduli personalizzati ed integrarli nelle diverse funzioni che richiedono PAM.

