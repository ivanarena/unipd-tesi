% !TEX encoding = UTF-8
% !TEX TS-program = pdflatex
% !TEX root = ../tesi.tex

%**************************************************************
\chapter{Introduzione}
\label{cap:introduzione}
%**************************************************************

\intro{In questo capitolo vengono descritti l'azienda ospitante ed il progetto dell'attività di stage curriculare.}


% \noindent Esempio di utilizzo di un termine nel glossario \\
% \gls{api}. \\

% \noindent Esempio di citazione in linea \\
% \cite{site:agile-manifesto}. \\

% \noindent Esempio di citazione nel pie' di pagina \\
% citazione\footcite{womak:lean-thinking} \\

%**************************************************************
\section{L'azienda}

Athesys Srl è un'azienda di consulenza informatica nata a Padova nel 2010 "dalla sinergia di affermati professionisti del settore IT", specializzata in ambito System Integration, Database Management, Sicurezza applicativa, Governance Cloud Platform, Hyperconvergenza e
Sviluppo Software in modalità Agile.

Athesys comprende la spin-off Monokee, fondata nel 2017 come soluzione cloud-based per la gestione dell'identità
e dell'accesso (IDaaS), la quale offre, come funzionalità principale, un sistema di \acrfull{sso} basato 
su diversi tipi di autenticazione, sia passwordless che tramite soluzioni di \acrfull{ssi}.

\vspace{20pt}
\begin{figure}[!h] 
    \centering 
    \includegraphics[width=0.4\columnwidth]{logo-athesys} 
    \caption{Logo di Athesys Srl}
\end{figure}

\begin{figure}[!h] 
    \centering 
    \includegraphics[width=0.4\columnwidth]{logo-monokee} 
    \caption{Logo di Monokee Srl}
\end{figure}
    

%**************************************************************
\section{Il progetto}

L'idea per l'attività di stage nasce proprio dall'esigenza dell'azienda di aumentare la portata di Monokee estendendo il 
relativo \acrshort{sso} anche a livello macchina, per poter, successivamente, configurare dei terminali
che possano gestire accesso e sessioni degli utenti Monokee già da sistema.

L'obiettivo del progetto del tirocinio era, dunque, era la ricerca e l'eventuale sviluppo di una soluzione che consentisse
di accedere a macchine UNIX Debian e \acrfull{rhel} tramite il proprio account Monokee in modo nativo, sfruttando l'infrastruttura
di Single Sign-On fornita dalla spin-off.

Il framework da utilizzare era FreeIPA, un gestore delle identità e degli accessi (Identity and Access Management, IAM)
gratuito ed open-source che combina tecnologie quali Linux, LDAP, MIT Kerberos, NTP, DNS ed SSSD e consta di un'interfaccia web
e di strumenti di amministrazione tramite command-line. 

L'attività, dalla durata totale di circa trecento ore, si è sviluppata inizialmente in un fase di ricerca e sperimentazione
con l'installazione del software FreeIPA su più macchine virtuali CentOS, RHEL e Ubuntu,
messe a disposizione dall'azienda.

La seconda fase è stata dedicata alla ricerca di un metodo che consentisse di effettuare l'autenticazione con il
proprio account Monokee su tali macchine; in tal senso, è stata approfondita la parte relativa a 
Linux PAM (Pluggable Authentication Modules) per studiare la possibilità dello sviluppo di un modulo aggiuntivo. 

In seguito a tale ricerca, ho deciso di optare per il sistema di autenticazione tramite Identity Provider esterno
messo a disposizione dall'applicativo di FreeIPA e di procedere, dunque, con la configurazione di un'applicazione
Monokee OAuth2 e di un provider OpenID Connect (OIDC) che fornissero gli end-point e l'infrastruttura 
necessaria alla comunicazione con il server di FreeIPA e la successiva implementazione degli stessi su di esso. 

Verificato il corretto funzionamento di questo sistema di autenticazione da \arcshort{cli}, ho proseguito cercando di implementare questo sistema anche tramite SSH fino al raggiungimento delle ore previste, tuttavia, senza successo.



%**************************************************************
% \section{Organizzazione del testo}

% \begin{description}
%     \item[{\hyperref[cap:processi-metodologie]{Il secondo capitolo}}] descrive ...
    
%     \item[{\hyperref[cap:descrizione-stage]{Il terzo capitolo}}] approfondisce ...
    
%     \item[{\hyperref[cap:analisi-requisiti]{Il quarto capitolo}}] approfondisce ...

%     \item[{\hyperref[cap:progettazione-codifica]{Il quinto capitolo}}] approfondisce ...
    
%     \item[{\hyperref[cap:verifica-validazione]{Il sesto capitolo}}] approfondisce ...
    
%     \item[{\hyperref[cap:conclusioni]{Nel settimo capitolo}}] descrive ...
% \end{description}
