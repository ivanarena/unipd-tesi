% !TEX encoding = UTF-8
% !TEX TS-program = pdflatex
% !TEX root = ../tesi.tex

%**************************************************************
% Sommario
%**************************************************************
\cleardoublepage
\phantomsection
\pdfbookmark{Abstract}{Abstract}
\begingroup
\let\clearpage\relax
\let\cleardoublepage\relax
\let\cleardoublepage\relax

\chapter*{Abstract}

Il presente documento descrive il lavoro svolto durante il periodo di stage, della durata di circa trecento ore, dal laureando \myName \space presso l'azienda \myAzienda. \\
    L'obiettivo principale dello stage era la ricerca e l'eventuale sviluppo di una soluzione per l'implementazione di un sistema di Single Sign-On (SSO) che permettesse, nativamente, l'autenticazione e la gestione della relativa sessione su una macchina UNIX Debian-like o RHEL-like tramite Monokee. \\
    Il framework da utilizzare era FreeIPA, un gestore delle identità e degli accessi (Identity and Access Management, IAM) gratuito ed open-source che combina tecnologie quali Linux, LDAP, MIT Kerberos, NTP, DNS ed SSSD e consta di un'interfaccia web user-friendly e di strumenti di amministrazione tramite command-line. \\
    L'attività si è sviluppata inizialmente in un fase di ricerca e sperimentazione con l'installazione del software FreeIPA su più macchine virtuali CentOS, RHEL e Ubuntu, messe a disposizione dall'azienda. \\
    La seconda fase è stata dedicata alla ricerca di un metodo che consentisse di effettuare l'autenticazione con il proprio account Monokee;
in tal senso, è stata approfondita la parte relativa a Unix PAM (Pluggable Authentication Modules) per studiare la possibilità dello sviluppo di un modulo aggiuntivo. \\
    In seguito a tale ricerca, si è deciso di optare per il sistema di autenticazione tramite Identity Provider esterno messa a disposizione dall'applicativo di FreeIPA
e di procedere, dunque, con la configurazione di un'applicazione Monokee OAuth2 e di un provider OpenID Connect (OIDC) che fornissero gli end-point e l'infrastruttura 
necessaria alla comunicazione con il server di FreeIPA e la successiva implementazione degli stessi su di esso. \\
    Verificato il corretto funzionamento di questo sistema di autenticazione da CLI (Command-Line Interface), si è proseguito cercando di implementare questo sistema anche tramite SSH fino al raggiungimento delle ore previste.

%\vfill
%
%\selectlanguage{english}
%\pdfbookmark{Abstract}{Abstract}
%\chapter*{Abstract}
%
%\selectlanguage{italian}

\endgroup			

\vfill

